\documentclass[bachelor, och, labwork]{shiza}
% параметр - тип обучения - одно из значений:
%    spec     - специальность
%    bachelor - бакалавриат (по умолчанию)
%    master   - магистратура
% параметр - форма обучения - одно из значений:
%    och   - очное (по умолчанию)
%    zaoch - заочное
% параметр - тип работы - одно из значений:
%    referat    - реферат
%    coursework - курсовая работа (по умолчанию)
%    diploma    - дипломная работа
%    pract      - отчет по практике
% параметр - включение шрифта
%    times    - включение шрифта Times New Roman (если установлен)
%               по умолчанию выключен
\usepackage{subfigure}
\usepackage{tikz,pgfplots}
\pgfplotsset{compat=1.5}
\usepackage{float}

%\usepackage{titlesec}
\setcounter{secnumdepth}{4}
%\titleformat{\paragraph}
%{\normalfont\normalsize}{\theparagraph}{1em}{}
%\titlespacing*{\paragraph}
%{35.5pt}{3.25ex plus 1ex minus .2ex}{1.5ex plus .2ex}

\titleformat{\paragraph}[block]
{\hspace{1.25cm}\normalfont}
{\theparagraph}{1ex}{}
\titlespacing{\paragraph}
{0cm}{2ex plus 1ex minus .2ex}{.4ex plus.2ex}

% --------------------------------------------------------------------------%


\usepackage[T2A]{fontenc}
\usepackage[utf8]{inputenc}
\usepackage{graphicx}
\graphicspath{ {./images/} }
\usepackage{tempora}

\usepackage[sort,compress]{cite}
\usepackage{amsmath}
\usepackage{amssymb}
\usepackage{amsthm}
\usepackage{fancyvrb}
\usepackage{listings}
\usepackage{listingsutf8}
\usepackage{longtable}
\usepackage{array}
\usepackage[english,russian]{babel}

\usepackage[colorlinks=false]{hyperref}
\usepackage{url}

\usepackage{underscore}
\usepackage{setspace}
\usepackage{indentfirst} 
\usepackage{mathtools}
\usepackage{amsfonts}
\usepackage{enumitem}
\usepackage{tikz}
\usepackage{verbatim}
\usepackage{minted}

\newcommand{\eqdef}{\stackrel {\rm def}{=}}
\newcommand{\specialcell}[2][c]{%
\begin{tabular}[#1]{@{}c@{}}#2\end{tabular}}

\renewcommand\theFancyVerbLine{\small\arabic{FancyVerbLine}}

\newtheorem{lem}{Лемма}

\begin{document}

% Кафедра (в родительном падеже)
\chair{теоретических основ компьютерной безопасности и криптографии}

% Тема работы
\title{Доказательство задачи на NP-полноту}

% Курс
\course{3}

% Группа
\group{331}

% Факультет (в родительном падеже) (по умолчанию "факультета КНиИТ")
\department{факультета КНиИТ}

% Специальность/направление код - наименование
%\napravlenie{09.03.04 "--- Программная инженерия}
%\napravlenie{010500 "--- Математическое обеспечение и администрирование информационных систем}
%\napravlenie{230100 "--- Информатика и вычислительная техника}
%\napravlenie{231000 "--- Программная инженерия}
\napravlenie{10.05.01 "--- Компьютерная безопасность}

% Для студентки. Для работы студента следующая команда не нужна.
% \studenttitle{Студентки}

% Фамилия, имя, отчество в родительном падеже
\author{Токарева Никиты Сергеевича}

% Заведующий кафедрой
% \chtitle{} % степень, звание
% \chname{}

%Научный руководитель (для реферата преподаватель проверяющий работу)
\satitle{доцент} %должность, степень, звание
\saname{А. Н. Гамова}

% Руководитель практики от организации (только для практики,
% для остальных типов работ не используется)
% \patitle{к.ф.-м.н.}
% \paname{С.~В.~Миронов}

% Семестр (только для практики, для остальных
% типов работ не используется)
%\term{8}

% Наименование практики (только для практики, для остальных
% типов работ не используется)
%\practtype{преддипломная}

% Продолжительность практики (количество недель) (только для практики,
% для остальных типов работ не используется)
%\duration{4}

% Даты начала и окончания практики (только для практики, для остальных
% типов работ не используется)
%\practStart{30.04.2019}
%\practFinish{27.05.2019}

% Год выполнения отчета
\date{2022}

\maketitle

% Включение нумерации рисунков, формул и таблиц по разделам
% (по умолчанию - нумерация сквозная)
% (допускается оба вида нумерации)
% \secNumbering

%-------------------------------------------------------------------------------------------

\tableofcontents

  
  \section{Доказательство NP полноты}

    NP-полная задача — в теории алгоритмов задача с ответом «да» или «нет» из класса NP, к 
  которой можно свести любую другую задачу из этого класса за полиномиальное время (то есть при 
  помощи операций, число которых не превышает некоторого полинома в зависимости от размера 
  исходных данных). Таким образом, NP-полные задачи образуют в некотором смысле подмножество 
  «типовых» задач в классе NP: если для какой-то из них найден «полиномиально быстрый» алгоритм 
  решения, то и любая другая задача из класса NP может быть решена так же «быстро».

  Чтобы показать NP полноту задачи необходимо доказать, что задача принадлежит классу NP, и она является NP-трудной.

  \subsection{Постановка задачи}

  Дано множество целых чисел $S$; Выяснить, можно ли разбить его на две части с равными суммами, то есть найти множество
  $A \subseteq S$, для которого $\sum_{x \in A} x = \sum_{x \in S \backslash A} x$. Покажите, что эта задача ($SPP$) является $NP$-полной.
  
  \subsection{Доказательство поставленной задачи}

  \begin{enumerate}
    \item Докажем, что задача $SPP$ принадлежит классу $NP$:
    
    Известно, что нам дано множество $S$. Пусть существует такое разбиение, что $S = A \cup \overline{A}$. Тогда алгоритм будет 
    выглядеть следующим образом:

    \begin{enumerate}
      \item Необходимо проверить, чтобы каждый элемент $x \in A$ и $\overline{x} \in \overline{A}$ принадлежал множеству $S$.
      \item Пусть $s_1 = 0$ и $s_2 = 0$.
      \item $\forall x \in A$ выполнить $s_1 = s_1 + x$.
      \item $\forall \overline{x} \in \overline{A}$ выполнить $s_2 = s_2 + \overline{x}$.
      \item Убедиться, что $s_1 = s_2$.
    \end{enumerate}

    Алгоритм занимает линейное время в размере набора чисел множества $S$.
    
    \item Докажем, что задача $SPP$ является $NP$-трудной:
    
    Известно что задача о сумме подмножеств (Subset sum problem) относится к классу $NP$: $SSP \in NP$.
    
    \textbf{Постановка задачи о сумме подмножеств:} дано множесвто $S$, содержащее $n$ целых чисел и целое число $s$.
    Требуется выяснить возможно ли выбрать подмножество $\overline{S} \subseteq S$ с суммой 
    $S$: $\exists \overline{S} \subseteq S: \sum_{s_i \in \overline{S}}s_i = s$.
  
    Тогда чтобы доказать $NP$-трудную задачу, необходимо произвести сведение $SSP$ к $SPP$. 
    Пусть в качестве входных данных дано множество $S$ и искомая сумма $t$. Тогда необходимо найти подможество $A \subset S$, сумма
    чисел которого равна $t$. Пусть $s$ будет суммой элементов множества $S$. Заметим, что $t = \frac{1}{2} \sum_{x \in A}x$.
    Тогда получим следующее разбиение $\overline{A} = A \cup \{s - 2t\}$ в $SPP$.

    Теперь покажем, что задача $SPP$ сводится к вычислению суммы подмножества.

    Рассмотрим подмножество $A$, сумма которых равна $t$, тогда остальные элементы множества $S$ (обозначим $S \backslash A = \overline{A}$) будут
    иметь сумму $s - t = d$. Предположим, что исходное разбиение $A^{'} = A \cup \{s - 2t\}$, сумма которого равна $t^{'}$.

    Справедливы следующие наблюдения:

    \begin{center}
      $d = s - t$
    
      $d - t = s - t - t$
    
      $t^{'} = t + (s - 2t)$
    
      $s - t = d$, т.е. суммы $A^{'}$ и $\overline{A}$ равны.
    
      \end{center}

      Следовательно, исходный набор можно разбить на два подмножества суммы (s - t) каждое. Таким образом, поставленная задача $SPP$ решается. 

      Теперь предположим, что существует разбиение равной суммы ($A$, $\overline{A}$) множества $S^{'} = S \cup {s - 2t}$. 
      Сумма каждого подмножества определяется как:

      \begin{center}
        $l = \frac{s + (s - 2t)}{2} = s - t$
      \end{center}
    
      Таким образом, $A$ является подмножеством $S$ с суммой, равной $t$.

      Сдедовательно, задача $SPP$ является $NP$-полной.

  \end{enumerate}


  % Чтобы показать, что любое множетво A является $NP$-полной, необходимо показать четыре вещи:
  % \begin{enumerate}
  %   \item существует недетерминированный алгоритм с полиномиальным временем, который решает $A$, т. е. $A \in NP$; 
  %   \item любая $NP$-полная задача $B$ может быть сведена к $A$;
  %   \item приведение $B$ к $A$ работает за полиномиальное время;
  %   \item исходная задача $A$ имеет решение тогда и только тогда, когда $B$ имеет решение.
  % \end{enumerate}

  % Теперь покажем, что задача $SP$ является $NP$-полной.
  
  % \textbf{1.} $SP \in NP$: Предположим, множество $S$ можно разбить на два подмножества, чтобы элементы этих множеств имели равные суммы.
  
  % \textbf{2.} Известно что задача о сумме подмножеств (Subset sum problem) относится к классу $NP$: $SSP \in NP$.
  % Произведем сведение $SSP$ к $SP$: Напомним, что $SSP$ определяется следующим образом: задано множество целых чисел $X$,
  % найти подмножество $Y \subseteq X$ такое, что члены множества $Y$ в сумме составляют ровно $t = \frac{1}{2} \sum_{x \in X}x$. Пусть $s$ будет суммой членов
  % множества $X$, т.е. $\sum_{x \in X}x = s$. Тогда получим следующее разбиение $X^{'} = X \cup \{s - 2t\}$ в $SP$.
  
  % \textbf{3.} Очевидно, что это сведение множества $B$ к множеству $A$ работает за полиномиальное время.
  
  % \textbf{4.} Докажем, что $\langle X, t \rangle \in SSP$ тогда и только тогда, когда $\langle X^{'} \rangle \in SP$. Примечание
  % что сумма элементов $X^{'}$ составляет $s + (s - 2t) = 2s - 2t$.

  % \textbf{Достаточность.} Если в множестве $X$ существует набор чисел, сумма которых равна $t$, то оставшиеся числа в множестве $X$ (обозначим это множество $Y$) равны числу $o = s - t$. Следовательно, 
  % существует разбиение $X^{'} = X \cup \{s - 2t\}$, сумма которого равна $t^{'}$.
  
  % \begin{center}
  % $o = s - t$

  % $o - t = s - t - t$

  % $t^{'} = t + (s - 2t)$

  % $s - t = o$, т.е. суммы $X^{'}$ и $Y$.

  % \end{center}
  % \textbf{Необходимость.} Предположим, что существует разбиение $X^{'}$ на два множества такое, что сумма по каждое множество есть $s - t$. Одно из 
  % этих множеств содержит $s - 2t$ элементов. Если удалить эти элементы, то получим количество чисел, сумма которых равна $t$, и все эти числа находятся в $X$.  
  
  
  % \textbf{Доказательство проблемы в NP}
  % Пусть $(A, S \backslash A) = (A, \overline{A})$ есть сертификат. Мы можем проверить сертификат за линейное время, суммируя 
  % элементы в каждом наборе и сравнивая результат.

  % \textbf{Доказательство проблемы в NP-сложно}
  % Мы покажем, что задача о множестве-разбиении является $NP$-трудной, показав:


  \begin{thebibliography}{3}
    \bibitem{1}
    Книга Томаса Кормена <<Алгоритмы>> / [Электронный ресурс] URL: https://e-maxx.ru/bookz/files/cormen.pdf (дата обращения 02.05.2022), Яз. рус.
    \bibitem{2}
    Статья <<$NP$-полнота задачи о сумме подмножества>> / [Электронный ресурс] URL: https://inlnk.ru/Ken0wA (дата обращения 02.05.2022), Яз. рус.
    \bibitem{3}
    Статья <<$NP$-полная задача>> / [Электронный ресурс] URL: https://clck.ru/gmbKK (дата обращения 05.05.2022), Яз. рус.
  \end{thebibliography}
\end{document}
